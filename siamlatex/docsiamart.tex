\documentclass[]{siamart}

%% Used for \set, used in an example below
\usepackage{braket,amsfonts}

%% Used in table example below
\usepackage{array,subfig}

%% Used for TIKZ example, shown in the "Figures" section below.
\usepackage{tikz}

%% ------------------------------------------------------------------
%% Items that need to be moved into the CLS file.
%% ------------------------------------------------------------------
\usepackage[capitalize,nameinlink]{cleveref}

\crefformat{section}{#2\S#1#3}
\crefrangeformat{section}{#3\S\S#1#4--#5#2#6}
\crefmultiformat{section}{#2\S\S#1#3}{ and #2#1#3}{, #2#1#3}{, and #2#1#3}
\crefrangemultiformat{section}{#3\S\S#1#4--#5#2#6}%
{ and #3#1#4--#5#2#6}{, #3#1#4--#5#2#6}{, and #3#1#4--#5#2#6}

\Crefformat{section}{#2Section~#1#3}
\Crefrangeformat{section}{#3Sections~#1#4--#5#2#6}
\Crefmultiformat{section}{#2Sections~#1#3}{ and #2#1#3}{, #2#1#3}{, and #2#1#3}
\Crefrangemultiformat{section}{#3Sections~#1#4--#5#2#6}%
{ and #3#1#4--#5#2#6}{, #3#1#4--#5#2#6}{, and #3#1#4--#5#2#6}

\crefformat{equation}{#2(#1)#3}
\crefrangeformat{equation}{(#3#1#4--#5#2#6)}
\crefmultiformat{equation}{#2(#1)#3}{ and #2(#1)#3}
{, #2(#1)#3}{, and #2(#1)#3}
\crefrangemultiformat{equation}{(#3#1#4--#5#2#6)}%
{ and (#3#1#4--#5#2#6)}{, (#3#1#4--#5#2#6)}{, and (#3#1#4--#5#2#6)}

\Crefformat{equation}{#2Equation~(#1)#3}
\Crefrangeformat{equation}{Equations~(#3#1#4--#5#2#6)}
\Crefmultiformat{equation}{Equations~(#2#1#3)}{ and (#2#1#3)}
{, (#2#1#3)}{, and (#2#1#3)}
\Crefrangemultiformat{equation}{Equations~(#3#1#4--#5#2#6)}%
{ and (#3#1#4--#5#2#6)}{, (#3#1#4--#5#2#6)}{, and (#3#1#4--#5#2#6)}


%% ------------------------------------------------------------------
%% End that need to be moved into the CLS file.
%% ------------------------------------------------------------------

%% ------------------------------------------------------------------
%% Macros for in-document examples. These are not meant to reused for
%% SIAM journal papers.
%% ------------------------------------------------------------------
%\usepackage{environ}
\usepackage{xspace}
\usepackage{bold-extra}
\usepackage[most]{tcolorbox}
\newcommand{\BibTeX}{{\scshape Bib}\TeX\xspace}
\newcounter{example}
\colorlet{texcscolor}{blue!50!black}
\colorlet{texemcolor}{red!70!black}

\lstdefinestyle{siamlatex}{%
  style=tcblatex,
  texcsstyle=*\color{texcscolor},
  texcsstyle=[2]\color{texemcolor},
  keywordstyle=[2]\color{texemcolor},
  moretexcs={cref,Cref},
}

\tcbset{%
  colframe=black!75!white!75,
  coltitle=white,
  colback=black!30!white!30, % left side
  colbacklower=white, % right side
  fonttitle=\bfseries,
  arc=0pt,outer arc=0pt,
  top=1pt,bottom=1pt,left=1mm,right=1mm,middle=1mm,boxsep=1mm,
  leftrule=0.3mm,rightrule=0.3mm,toprule=0.3mm,bottomrule=0.3mm,
  listing options={style=siamlatex}
}

\newtcblisting[use counter=example]{example}[2][]{%
  title={Example~\thetcbcounter: #2},#1}

\newtcbinputlisting[use counter=example]{\examplefile}[3][]{%
  title={Example~\thetcbcounter: #2},listing file={#3},#1}

\DeclareTotalTCBox{\code}{ v O{} }
{ fontupper=\ttfamily\color{texemcolor},
  nobeforeafter,
  tcbox raise base,
  colback=white,colframe=white,
  top=0pt,bottom=0pt,left=0mm,right=0mm,
  leftrule=0pt,rightrule=0pt,toprule=0mm,bottomrule=0mm,
  boxsep=0.5mm,
  #2}{#1}

\newcommand{\Comment}[2][To do]{\textcolor{green!60!black}{\textbf{#1:} #2}}

%% ------------------------------------------------------------------
%% End of macros for in-document examples. 
%% ------------------------------------------------------------------

\begin{document}

%% ------------------------------------------------------------------
%% HEADER
%% ------------------------------------------------------------------
\begin{tcbverbatimwrite}{\jobname_header.tex}
\title{Guide to Using SIAM's \LaTeX\ Style%
  \thanks{Acknowledgments such as funding go here.}}

\subtitle{\today} 

\author{Dianne Doe%
  \thanks{Imagination Corp., Chicago, IL, \email{ddoe@imag.com}.}%
  \and
  Paul T. Frank%
  \thanks{Department of Mathematics, Fictional University, Boise, ID.}
  \and
  Jane E. Smith%
  \footnotemark[3] % Reuse the 3rd "thanks" footnote
  \footnotemark[4] % Define contents below
}

\maketitle

% Need to briefly change \thefootnote for inserting text explicitly
% (rather than via \thanks)
\renewcommand{\thefootnote}{\fnsymbol{footnote}}
\footnotetext[4]{These are example author names.}
\renewcommand{\thefootnote}{\arabic{footnote}}
\end{tcbverbatimwrite}
\input{\jobname_header.tex}
%% ------------------------------------------------------------------
%% END HEADER
%% ------------------------------------------------------------------

%% ------------------------------------------------------------------
%% MARKBOTH
%% ------------------------------------------------------------------
\begin{tcbverbatimwrite}{\jobname_markboth.tex}
\pagestyle{myheadings}
\thispagestyle{plain}
\markboth{Dianne Doe, Paul T. Frank, and Jane E. Smith}%
{Guide to Using  SIAM'S \LaTeX\ Style}
\end{tcbverbatimwrite}
\input{\jobname_markboth.tex}
%% ------------------------------------------------------------------
%% END MARKBOTH
%% ------------------------------------------------------------------



%% ------------------------------------------------------------------
%% ABSTRACT
%% ------------------------------------------------------------------
\begin{tcbverbatimwrite}{\jobname_abstract.tex}
%\newcommand{\BibTeX}{{\scshape Bib}\TeX\xspace} <- Preamble
\begin{abstract}
  Documentation is given for use of the SIAM \LaTeX\ macros.
  Instructions and suggestions for compliance with SIAM style
  standards are also included. Familiarity with standard \LaTeX\
  commands is assumed.
\end{abstract}

\begin{keywords}
  \LaTeX, \BibTeX, SIAM Journals 
\end{keywords}

\begin{AMS}
\end{AMS}
\end{tcbverbatimwrite}
\input{\jobname_abstract.tex}
%% ------------------------------------------------------------------
%% END HEADER
%% ------------------------------------------------------------------

\section{Introduction}
\label{sec:intro}

This file is documentation for the SIAM \LaTeX\ style, including how
to typeset the main document, the \BibTeX\ file, and any supplementary
material. More information
about SIAM's editorial style can be found in the style manual, available
at \url{http://www.siam.org/journals/pdf/stylemanual.pdf}.
%
The SIAM latex files can be found at
\url{http://www.siam.org/journals/auth-info.php}. The files are that
are distributed are given below.
\begin{itemize}
\item \texttt{siamart.cls} (required): Main \LaTeX\ class file.
\item \texttt{siamplain.bst} (required): Bibliographic style file for \BibTeX.
\item \texttt{docsiamart.tex}: Produces this documentation.
\item \texttt{docsiambib.bib}: Example \BibTeX\ database.
\item \texttt{docsiamsupp.tex}: Supplemental file example and documentation.
\end{itemize}
%
The outline of a SIAM \LaTeX\ article is shown in \cref{ex:outline}.

\begin{example}[label={ex:outline},listing only,%
  listing options={style=siamlatex,{morekeywords=[2]{siamart}},}]%
  {Document outline}
\documentclass{siamart}

% Packages and macros definitions go here.

\begin{document}

% Front matter goes here: title, authors, abstract, etc.
% Main body goes here.
% Appendices goes here (optional).
% Acknowledgements go here (optional).
% Bibliography goes here.

\end{document}
\end{example}

Class options can be included in the bracketed argument of the
command, separated by commas.
By default, lines which extend past the
margin will have black boxes next to them to help authors identify
lines that they need to fix, by re-writing or inserting
breaks. Enabling the \code{final} option
turns these boxes off, so that very small margin breaks which are not
noticible will not cause boxes to be generated.
Use the \code{review} option to create line numbers, useful for papers
submitted to SIAM for review.


\section{Front matter}
\label{sec:front}
\Comment{We should change this to set pdftitle and pdfauthor, and then also use these in setting the page headers with markboth.}
The title and author parts are formatted using the standard
\code{\title}, \code{\author}, and \code{\maketitle} commands as
described in Lamport \cite{La86}.
%
If there is more than one author, the authors should be separated by
the \code{\and} command.  
%
The addresses and support acknowledgments are added via
\code{\thanks}. Each author's thanks should specify their address.
The support acknowledgment should be put in a thanks for the title,
unless specific support needs to be specified for individual authors,
in which case it should follow the author address or be in a separate
footnote.  
%
The header for this file was produced by the code in \cref{ex:header}, including
examples of various footnote specifications.
%
\Comment[To fix]{The fourth footnote on the cover page is not aligned with the other footnotes.}

\examplefile[label={ex:header},listing only,%
listing options={style=siamlatex,%
deletetexcs={and,thanks,title,author},%
{moretexcs=[2]{and,thanks,title,author,maketitle}}}
]{Title and authors}{\jobname_header.tex}


\Cref{ex:markboth} shows how to specify the page headings, with
the authors' names and the title (possibly shortened to fit).

\examplefile[label={ex:markboth},listing only,%
listing options={style=siamlatex,%
deletetexcs={pagestyle,thispagestyle,markboth},%
{moretexcs=[2]{pagestyle,thispagestyle,markboth}}}
]{Page headers}{\jobname_markboth.tex}

Following the author and title is the abstract, key words listing, and AMS subject
classifications, designated using the \code{abstract}, \code{keywords}, and \code{AMS}
environments. 
Authors are responsible for providing AMS numbers which can be found
on the AMS web site \cite{AMSMSC2010}. The abstract, keywords, and AMS subject classifications
for this document were specified in \cref{ex:abstract}.

\examplefile[label={ex:abstract},listing only,%
listing options={style=siamlatex,%
{morekeywords=[2]{abstract,keywords,AMS}}}
]{Abstract, keywords, and AMS classifications}{\jobname_abstract.tex}


\section{Sections and cross-referencing}
\label{sec:sec}
Sections are denoted using standard \LaTeX\ section commands, i.e., \code{\section}, \code{\subsection}, etc. The appendices are defined the same way except that the first one is preceded by the \code{\appendix} command.
The acknowledgment section is preceeded by \code{\section*{Acknowledgments}}; it comes after any appendices and before the bibliography.

SIAM uses the \texttt{cleveref} package for cross-referencing, including
customizations to adhere to SIAM's style guidelines. 
%
The macros automatically determine the proper way to format standard
references, including the name of the reference and the hyperlink.
%
Use \code{\Cref} for a reference at the beginning of a sentence and
\code{\cref} otherwise.
%
A label for a section should always begin with \texttt{sec}.
%
\Cref{ex:secref} shows how to reference sections.


\Comment{Hyperlinks are currently black, even though they are enabled. We may want to turn on some colors for testing purposes and/or to try different options.}

\Comment{For some reason, the SIAM style does not have PDF
  bookmarks. Let's fix that.}

\Comment{Explain how to reference section is the supplement.}

\begin{example}[label={ex:secref},lefthand ratio=0.4,bicolor,sidebyside,%
listing options={style=siamlatex,basicstyle=\ttfamily\scriptsize,%
deletetexcs={cref,Cref},{moretexcs=[2]{cref,Cref}}}]
{Right and wrong ways to reference a section}
Inside a sentence\dots\\
Single: \cref{sec:intro}\\
Range: \cref{sec:intro,sec:front,%
sec:sec}\\
Multiple: \cref{sec:intro,sec:sec,%
sec:tab,sec:math,sec:thm}\\
Appendix: \cref{sec:changes}\\

Beginning of a sentence\dots\\
Single: \Cref{sec:intro}\\
Range: \Cref{sec:intro,sec:front,%
sec:sec}\\
Multiple: \Cref{sec:intro,sec:sec,%
sec:tab,sec:math,sec:thm}\\
Appendix: \Cref{sec:changes}\\


Just don't do it this way\dots\\
Section~\ref{sec:intro}  
\end{example}

\section{Math and equations}
\label{sec:math}
Here we show some example equations, with numbering, and examples of
referencing the equations.
The SIAM \LaTeX\ class adds the following macros by default: 
\code{\const},
\code{\diag},
\code{\grad},
\code{\Range},
\code{\rank},
\code{\supp}.
%
This have the effect of rendering the item as a \texttt{mathop}. 
\Cref{ex:matrices,ex:range,ex:ml,ex:aligned,ex:foo} use many of the features of the package \code{amsmath} and examples
from \cite{MiGo04}.


\begin{example}[label={ex:matrices},bicolor]{Creating matrices}
\Cref{eq:matrices} shows some environments for making nice matrices.
\begin{equation}\label{eq:matrices}
  S=\begin{bmatrix}1&0\\0&0\end{bmatrix}
  \quad\text{and}\quad
  C=\begin{pmatrix}1&1&0\\1&1&0\\0&0&0\end{pmatrix}.
\end{equation}
\end{example}

\begin{example}[label={ex:range},bicolor,%
listing options={style=siamlatex,%
{moretexcs=[2]{Range}}}
]{Using SIAM-defined macros}
% \usepackage{braket,amsfonts} <- Preamble
We use a special SIAM macro in \cref{eq:range}.
\begin{equation}\label{eq:range}
  \Range(A) = \set{ y \in \mathbb{R}^n | y = Ax }
\end{equation}
\end{example}

\begin{example}[label={ex:ml},bicolor,sidebyside,%
lefthand ratio=0.5]{Equation split across lines}
\begin{multline}\label{eq:ml}
(a+b)^2 +c^2 + d^2 \\
= a^2 + 2ab + b^2 +c^2 + d^2
\end{multline}
\end{example}

\begin{example}[label={ex:aligned},bicolor,sidebyside]{Aligned equations}
\begin{align}
x^2 + y^2 &= z^2 \label{eq:aa}\\
x^3 + y^3 &\geq z^3 \nonumber \\
x^3 + y^3 &< z^3 \label{eq:bb}
\end{align}  
\end{example}

\Comment[To fix]{May need to tell cleveref how to reference subequations.}

\begin{example}[label={ex:foo},bicolor,sidebyside,lefthand ratio=.7]{Subequations}
\begin{subequations} \label{eq:foo}
\begin{align} f &= g \label{subeq:a} \\
  f' &= g' \label{subeq:b} \\
  \mathcal{L}f &= \mathcal{L}g \label{subeq:c}
\end{align}
\end{subequations}
\end{example}

\Cref{ex:eqref} shows how to reference equations using \code{\cref}.

\begin{example}[label={ex:eqref},bicolor]%
{Right ways to reference equations}
Equations are automatically referenced correctly as a single equation
\cref{eq:range}, a range of equations \cref{eq:matrices,eq:range,eq:ml}, 
or an arbitrary set of equations: \cref{eq:matrices,eq:ml,eq:aa,eq:bb}.

\par\smallskip
Be sure to use the correct version of the beginning of a sentence so that 
the word ``Equation'' is inserted first for a single: \Cref{eq:range},
range: \Cref{eq:matrices,eq:range,eq:ml} or multiple: 
\Cref{eq:matrices,eq:ml,eq:aa,eq:bb}.

\par\smallskip
\Cref{eq:foo} is an example of using the subequations environment, with suequations \cref{subeq:a,subeq:b,subeq:c}. 
\end{example}

\section{Theorem-like environments}
\label{sec:thm}

\Comment{Need to show what theorem-like environments exist, how to create new ones, and how to reference them.}

\section{Lists}

\Comment{SIAM prefers special list styles. These should be achievable via options to the enumitem package.}

\section{Tables}
\label{sec:tab}

\Cref{ex:table} show the code to generate \cref{tab:KoMa14}. This
example uses subfloats via the \texttt{subfig} package, as well as
special column options from the \texttt{array} package.

\begin{tcbverbatimwrite}{\jobname_table.tex}
%\usepackage{array,subfig} <- Preamble
\newcolumntype{R}{>{$}r<{$}}  %
\newcolumntype{V}[1]{>{[\;}*{#1}{R@{\;\;}}R<{\;]}} %
\begin{table}[htbp]
  \caption{Example table adapted from Kolda and Mayo \cite{KoMa14}.}
  \label{tab:KoMa14}
  \centering
  \subfloat[$\beta=1$]{
    \begin{tabular}{|c|R|V{3}|c|r@{\,$\pm$\,}l|} \hline
occ. & \multicolumn{1}{c|}{$\lambda$} & \multicolumn{4}{c|}{$\mathbf{x}$} & 
fevals &  \multicolumn{2}{c|}{time (sec.)}\\ \hline
 718 & 11.3476 &  0.5544 &  0.3155 &  1.2018 &  0.0977 &   45 & 0.17 & 0.06 \\ \hline 
 134 &  3.7394 &  0.2642 & -1.1056 &  0.2657 & -0.3160 &   31 & 0.12 & 0.05 \\ \hline 
 144 &  2.9979 &  1.0008 &  0.4969 & -0.0212 & -0.4817 &   31 & 0.12 & 0.05 \\ \hline 
   4 & \multicolumn{6}{c|}{\emph{--- Failed to converge ---}} & 0.21 & 0.10 \\ \hline 
 \end{tabular}}

  \subfloat[$\beta=-1$]{
    \begin{tabular}{|c|R|V{3}|c|r@{\,$\pm$\,}l|} \hline
occ. & \multicolumn{1}{c|}{$\lambda$} & \multicolumn{4}{c|}{$\mathbf{x}$} & 
fevals &  \multicolumn{2}{c|}{time (sec.)}\\ \hline
  72 & -1.1507 &  0.2291 &  0.6444 &  0.3540 & -0.8990 &   34 & 0.14 & 0.06 \\ \hline 
 150 & -3.2777 &  0.8349 & -0.7603 & -0.3532 & -0.2635 &   33 & 0.14 & 0.07 \\ \hline 
 148 & -3.5998 &  1.0486 &  0.6046 &  0.3736 &  0.3971 &   41 & 0.16 & 0.08 \\ \hline 
 624 & -6.3985 &  0.1003 &  0.1840 &  0.5305 &  1.2438 &   48 & 0.19 & 0.08 \\ \hline 
   4 & \multicolumn{6}{c|}{\emph{--- Converged to wrong solution ---}} & 0.10 & 0.11 \\ \hline 
   2 & \multicolumn{6}{c|}{\emph{--- Failed to converge ---}} & 0.23 & 0.02 \\ \hline 
 \end{tabular}}
\end{table}
\end{tcbverbatimwrite}

\examplefile[float=htpb,label={ex:table},listing only, listing options={%
  style=siamlatex,basicstyle=\ttfamily\scriptsize}]%
{Example table with subtables}{\jobname_table.tex}

\input{\jobname_table.tex}

\section{Figures}
\label{sec:fig}
\Comment{Need to add Tikz and epstopdf examples in this section.}


\section{Supplemental material}

\Comment{Explain how to do supplementary material, including \LaTeX\ files as well as multimedia.}

\section{Bibliography}

\Comment{Need to add some discussion here, including new fields. Show the code that is used the generate the bibliography.}

The \BibTeX for \cite{KoMa14} is shown below in \cref{ex:article}; observe the new \code{doi} field which produces a hyperlink in the citation.

\examplefile[float=htbp,label=ex:article,listing only,%
listing options={style=siamlatex,firstline=1,lastline=12,{morekeywords=[2]{doi}}}
]{Example article in \BibTeX}{docsiambib.bib}

\Comment{Tammy needs to add many more examples to demonstrate the new fields:
ArXiv example: \cite{PeKoPi14},
PubMed example: \cite{ZhDuYanQi15}
}

\section{Other hyperlinks}
The SIAM \LaTeX\ class adds a command for specifying an email address with a hyperlink:
% \begin{coderunb}{Email hyperlink}
% To obtain the data, send email 
% to \email{datasets@imag.com}. 
% \end{coderunb}

\appendix
\section{Changes}
\label{sec:changes}
The new SIAM styles includes the following significant changes as compared to older versions:
\begin{itemize}
\item Removed uppercase on title.
\end{itemize}

\section*{Acknowledgments}
Acknowledgements go here.
\Comment{Something is broken. There should not be an ``Appendix'' starting this, and the title is missing.}

\bibliographystyle{siamplain}
\bibliography{docsiambib}

\bigskip

\Comment{This files creates a bunch of extra files as it compiles. Not sure how to remove them.}

\end{document}
